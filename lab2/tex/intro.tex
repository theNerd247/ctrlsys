\documentclass[main.tex]{subfile}

\begin{document}
\section{Introduction} 
\label{sec:introduction}

The PID controller is a popular control system that is cheap and simple to
create and maintain in inductry. PID controllers allow the engineer to develop
control systems that are not only stable and reach the given input signal, but
can also control how fast the system responds and base its output on the
system's current output (in otherwords a PID controller is a feedback controller.)

The idea behind a PID controller is to optimize the performance of a control
signal using three properties of the error signal $e(t)$ (the difference between the
set point of the controller and the current controller output - $r(t)-c(t)$): 

\begin{itemize}
	\item proportional: $K_p e(t)$
	\item integral: $K_i \int_0^t e(t) \del{t}$
	\item derivative $K_d D[e(t)]$
\end{itemize}

A PID control system uses an intermediate controller between the error
computation and the system plant (or controller). This intermediate layer takes
as its input the error of the system and generates a control signal for the
system plant:

\begin{align}
	\label{eq:gc_t}
	G_c(t) = K_p e(t) + K_i \int_0^t e(t) \del{t} + K_d D[e(t)]
\end{align}

In s-space \eqref{gc_t} is: 

\begin{align}
	\label{eq:gc_s}
	G_c(s) &= \frac{U(s)}{E(s)}
	\\     &= K_p + \frac{K_i}{s} + K_ds
\end{align}

Thus, given a system plant, $G(s)$ and a unity feedback we have the entire
system controller $T(s)$: 

\begin{align}
	T(s) &= \frac{C(s)}{R(s)}
	\\   &= \frac{G_cG}{1+G_cG}
\end{align}


Because \eqref{gc_s} can be viewed as the superposition of three different
controllers (the proportional, integral, and derivative) for the control signal
the affects of each constant ($K_p$,$K_i$,$K_d$) can be analyzed or synthesized
separately in order to obtain the optimum performance for a given system. And
thus by first isolating a single PID constant in the system and generating $N$
number of control systems (each with a variation of that isolated PID constant)
we can observe the affects of that variable on the system.

% section introduction (end)

\end{document}
